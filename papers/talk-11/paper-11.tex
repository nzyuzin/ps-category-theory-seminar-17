\def\pathToRoot{../../}\input{\pathToRoot headers/header_lecturenotes}

\usepackage{todonotes}
\usepackage{semantic}

\DeclareMathOperator{\core}{core}

% Theorem Environments
\newtheorem{definition}{Definition}
\newtheorem{example}{Example}
\newtheorem{lemma}{Lemma}

% Title page
\title{Presheaf Model for Dependent Types}

\author{Nikita Ziuzin}

\begin{document}

\maketitle

\section*{Dependent types}

Dependent types are types that unlike simple types like functions and products
may depend on or vary with values.  A classical example of a dependent type is
a type of lists $List_{A}(N)$ of a given length $N : \NN$ over some
type $A$. The elements of such type are $Nil_{A} : List(0)$ and
$Cons_{A}(x, l) : List(N+1)$, where $x : {A}$ and $l : List_{A}(N)$.
Now let us consider a function that takes $n : \NN$ and returns a list of
length $n$ containing $0$ in all entries.  This function would have the type
$\Pi n : \NN.List_\NN(n)$, a type of functions for which the type
of the return value depends on the argument.  The important point of the
dependent typing is that it reveals more information about the function.
Another example is an exception-free head function that returns the first
element of the list:
\[
  hd: \Pi x : \mathbb{N}.List_{A}(Suc(x)) \to A
\]
The type of this function prevents it from being applied to an empty list,
showing clearly that the function is defined only for non empty lists.

\section*{Formalization}

To formalize dependent types we build a system that allows us to make
judgements of the form $a : A$ (term $a$ has type $A$) and $A~type$ ($A$ is a
type). We will also want a notion of definitional equality to be built into the
system, for example we want to consider $0 : \NN$ and $0 + 0 : \NN$ to be
(definitionally) equal terms and $List_A(0)$, $List_A(0 + 0)$ as
(definitionally) equal types: if $l : List_A(0)$ then also $l : List_A(0 + 0)$.
This leads to two more forms of the judgements: $a = b : A$ ($a$ and $b$ are
definitionally equal terms of type $A$) and $A = B~type$ ($A$ and $B$ are
definitionally equal types). Finally, we must know types of the free variables
occurring in a judgement: we cannot assert $x + y : \NN$ unless we know $x :
\NN$ and $y : \NN$. Since such declarations may depend on each other like in $x
: \NN$, $y : List_\NN(x)$ it is convenient to make all judgements relative to a
list of variable declarations including at least the free variables occurring
inside the judgement. Such lists are called contexts and sometimes also type
assignments.

Intuitively, a context $x_1 : A_1, \dots, x_n : A_n$ is well-formed if each
$A_i$ is a type in the context $x_1 : A_1, \dots, x_{n-1} : A_{n-1}$ and all
the variables $x_i$ are pairwise distinct. Thus we arrive to six kinds of
judgements:

\begin{align*}
  &\vdash \Gamma~ctxt && \text{$\Gamma$ is a valid context} \\
  &\Gamma \vdash A~type && \text{$A$ is a type in $\Gamma$} \\
  &\Gamma \vdash a : A && \text{$a$ is a term of type $A$ in context $\Gamma$} \\
  &\vdash \Gamma = \Delta && \text{$\Gamma$ and $\Delta$ are definitionally equal contexts} \\
  &\Gamma \vdash A = B && \text{$A$ and $B$ are definitionally equal types in context $\Gamma$} \\
  &\Gamma \vdash a = b : A && \text{$a$ and $b$ are definitionally equal terms of type $A$ in context $\Gamma$}
\end{align*}

We also have the following structural rules:
\begin{itemize}
  \item The empty context construction, context extension and context equality
    rules
    \[
      \inference[C-Emp]{}{\vdash \diamond~ctxt}
      \qquad
      \inference[C-Ext]{\Gamma \vdash A~type}{\vdash \Gamma, x : A~ctxt}
    \]
    \[
      \inference[C-Ext-Eq]{\vdash \Gamma = \Delta~ctxt \qquad \Gamma \vdash A = B~type}{\vdash \Gamma, x:A = \Delta, y:B~ctxt}
    \]
    where $x$ and $y$ are assumed to be fresh variables.
  \item The variable rule
    \[
      \inference[Var]{\vdash \Gamma, x:A,\Delta~ctxt}{\Gamma,x:A,\Delta \vdash x:A}
    \]
  \item Rules expressing that definitional equality is an equivalence relation:
    \[
      \inference[C-Eq-R]{\vdash \Gamma~ctxt}{\vdash \Gamma = \Gamma~ctxt}
    \]
    \[
      \inference[C-Eq-S]{\vdash \Gamma = \Delta~ctxt}{\vdash \Delta = \Gamma~ctxt}
    \]
    \[
      \inference[C-Eq-T]{\vdash \Gamma = \Delta~ctxt \qquad \vdash \Delta = \Theta~ctxt}{\vdash \Gamma = \Theta~ctxt}
    \]
    with analogous rules for terms and types (omitted).
  \item Rules relating typing and definitional equality:
    \[
      \inference[Tm-Conv]{\Gamma \vdash a : A \qquad \vdash \Gamma = \Delta~ctxt \qquad \Gamma \vdash A = B~type}{\Delta \vdash a : B}
    \]
    \[
      \inference[Ty-Conv]{\vdash \Gamma = \Delta~ctxt \qquad \Gamma \vdash A~type}{\Delta \vdash A~type}
    \]
  \item And rules for weakening and substitution, where $\mathcal{J}$ ranges over judgements of the form $a : A, A~type, a = b: A, A = B~type$:
    \[
      \inference[Weak]{\Gamma, \Delta \vdash \mathcal{J} \qquad \Gamma \vdash C~type}{\Gamma, x : C, \Delta \vdash \mathcal{J}}
    \]
    \[
      \inference[Subst]{\Gamma, x: C, \Delta \vdash \mathcal{J} \qquad \Gamma \vdash c: C}{\Gamma, \Delta[c/x] \vdash \mathcal{J}[c/x]}
    \]
    Here $\mathcal{J}[c/x]$ (and $\Delta[c/x]$) denotes the capture-free
    substitution of $x$ for $c$ in $\mathcal{J}$. This means that bound
    variables in $\mathcal{J}$ are systematically renamed to prevent any free
    variables in $c$ from becoming bound in $\mathcal{J}[c/x]$. The contexts,
    types, and terms will be considered equal if they agree up to names of
    bound variables and we will assume existence of capture-free substitution
    function on these equivalence classes.
\end{itemize}

\section*{Type formers}

Type and term formers are introduced by formation, introduction, elimination,
and equality rules. We present several type formers to give an example of how
they may be constructed.

\subsection*{Dependent product}

Dependent product (also known as dependent function space or $\Pi$-type)
corresponds to set-theoretic notion of cartesian product over a family of sets
$\Pi_{i \in I}B_i$ which has as elements functions mapping an index $i$ to an
element of a corresponding set $B_i$. In type theory this is expressed as
follows:

\[
  \inference[$\Pi$-F]{\Gamma \vdash A~type \qquad \Gamma, x:A \vdash B~type}{\Gamma \vdash \Pi x:A. B~type}
\]

\[
  \inference[$\Pi$-Eq]{\Gamma \vdash A_1 = A_2 ~type \qquad \Gamma, x:A \vdash B_1 = B_2 ~type}{\Gamma \vdash \Pi x:A_1. B_1 = \Pi x:A_2. B_2~type}
\]

The first rule expresses that a dependent product consists of a type $A$ (that
possibly depends on other types in $\Gamma$) and a type depending on $A$ (and
$\Gamma$), namely $B$. The rule $\Pi$-Eq expresses that the definitional
equality is respected by the $\Pi$-former. Note that the variable $x$ becomes
bound in $\Pi x:A. B$.

To form the elements of the $\Pi$-type we have the introduction rules with
associated congruence rule:

\[
  \inference[$\Pi$-I]{\Gamma, x:A \vdash b : B}{\Gamma \vdash \lambda x : A. b : \Pi x:A. B}
\]

\[
  \inference[$\Pi$-I-Eq]{\Gamma, x:A_1 \vdash b_1 = b_2 : B_1 \qquad \Gamma A_1 = A_2~type \qquad \Gamma x: A_1 \vdash B_1 = B_2~type}{\Gamma \vdash \lambda x : A_1. b^{B_1} = \lambda x : A_2. b^{B_2} : \Pi x:A_1. B_1}
\]

So to give an element of $\Pi x:A. B$ one must give an element of $B[x]$ in the
presence of a variable $x$ of type $A$. The congruence rule $\Pi$-I-Eq
expresses that definitional equality preserves $\Pi$-introduction. In the
following we will refrain from writing down the congruence rules explicitly,
having them present implicitly for every type and term formers we introduce.

Elements of $\Pi$-type are consumed using application:
\[
  \inference[$\Pi$-E]{\Gamma \vdash u : \Pi x:A. B \qquad \Gamma \vdash a : A}
  {\Gamma \vdash App_{[x:A]B}(u, a): B[a / x]}
\]

Square brackets in the typing annotation $[x : A] B$ here mean that the
variable $x$ is bound in $B$.

The following rule gives definitional equality to applications --- applying a
function $\lambda x : A. b^B$ to a term $a : A$ results in $b$ with $x$
replaced by $a$:

\[
  \inference[$\Pi$-C]{\Gamma \vdash \lambda x : A. b : \Pi x:A. B \qquad \Gamma \vdash a : A}
  {\Gamma \vdash App_{[x:A]B}(\lambda x : A. b, a) = b[a / x]: B [a / x]}
\]

Note that the substitution plays more prominent role here than in simply typed
lambda calculus: substitution is needed to formulate not only the term
conversion rules, but also the typing rules.

\subsection*{Dependent sum}

The next type former we consider is the $\Sigma$-type, or dependent sum,
corresponding to disjoint union in set theory. If we are given a family of sets
$(B_i)_{i \in I}$ we can form the set $\Sigma_{i \in I} B_i := \{ (i, b)~|~i
\in I \land b \in B_i\}$ whose elements consist of an index $i$ and an element
of the corresponding set $B_i$. In type theory the corresponding formation and
introduction rules look as follows.
\[
  \inference[$\Sigma$-F]{\Gamma \vdash A \text{ type}\\ \Gamma, x:A \vdash B \text{ type}}{\Gamma \vdash \Sigma x : A. B \text{ type}}
\]
\[
  \inference[$\Sigma$-I]{\Gamma \vdash a : A\\ \Gamma\vdash b: B[a / x]}{\Gamma Pair_{[x:A]B} (a,b) : \Sigma x : A. B}
\]

Elimination rule:
\[
  \inference[$\Sigma$-E]{\Gamma, z : \Sigma x: A. B \vdash C \text{ type}\\ \Gamma, x: A, y: B \vdash c : C[Pair_{[x:A]B}(x, y)/z]\\ \Gamma \vdash u : \Sigma x: A. B}
  {R^\Sigma_{[z : \Sigma x: A. B]C}([x : A, y: B]c, u): C[u/z]}
\]

And definitional equality:
\[
  \inference{\Gamma\vdash R^\Sigma_{[z : \Sigma x: A. B]C}([x : A, y: B]c, Pair_{[x:A]B}(a,b)): C[Pair_{[x:A]B}/z]}
  {\Gamma\vdash R^\Sigma_{[z : \Sigma x: A. B]C}([x : A, y: B]c, Pair_{[x:A]B}(a,b)) = c[a/x,b/y] : C [Pair_{[x:A]B}/z}
\]

The projection functions for the dependent sum can be defined using these
rules:
\[
  \mathbf{p}(a) := R^\Sigma_{[z : \Sigma x: A. B]A}([x : A, y: B]x, a) : A
\]
for the first projection and
\[
  \mathbf{q}(a) := R^\Sigma_{[z : \Sigma x: A. B]B[z.1/x]}([x : A, y: B]y, a) : B[\mathbf{p}(a)/x]
\]
for the second projection, assuming $\Gamma \vdash A~type$, $\Gamma, x:A \vdash
B~type$, and $\Gamma \vdash a : \Sigma x: A. B$.

The important special cases of $\Pi x: A. B$ and $\Sigma x : A. B$ arise when
$B$ does not actually depend on $A$. The we write
\[
  A \to B := \Pi x: A. B
\]
and
\[
  A \times B := \Sigma x: A. B
\]
indicating that in this cases $\Pi$- and $\Sigma$-types correspond to ordinary
non-dependent function and cartesian product types respectively.

\section*{Categories with Families}

We will use categories with families (CwF) to model dependent types. First
consider what is a category of families. The category of families of sets,
$\fcat{Fam}$, has as objects $(A, B)$ where $A$ is a set and $B = (B_a~|~a \in
A)$ is an $A$-indexed family of sets $B_a$, $a \in A$. A morphism $(A, B) \to
(A', B')$ is given by a pair $(f, g)$ where $f: A \to A'$ and $g$ is an
$A$-indexed famiy of functions such that $g_a: B_a \to B_{fa}'$.

A \emph{category with families} is given by ($\cat{C}$,$F$) where:

\begin{itemize}
  \item[$\bullet$] $\cat{C}$ is a category, objects: $\Gamma, \Delta, \dots$ are
    \emph{contexts}, morphisms:  $\sigma, \tau, \dots$ are
    \emph{substitutions} or context morphisms

  \item[$\bullet$] A terminal object $1$ in $\cat{C}$ is the empty context ($\diamond$).

  \item[$\bullet$] $F$ is a functor $F\from \cat{C}^\op \to \fcat{Fam}$; For a
    context $\Gamma$ we write $Ty_F(\Gamma)$ for the indexing set of the family
    $F(\Gamma)$ and its family as $(Ter_F(\Gamma;A) | A \in Ty_F(\Gamma))$;
    $\Gamma \vdash A~type$ means that $A \in Ty_F(\Gamma)$ and $\Gamma \vdash a
    : A$ means $a \in Ter_F(\Gamma;A)$. For a context morphism $\sigma \from
    \Delta \to \Gamma$ the morphism $F(\sigma)$ acts on types $\Gamma \vdash
    A~type$ as $A\sigma$ and on terms $\Gamma \vdash a : A$ as $a\sigma$. The
    following equations also hold:

    \[
      A1=A \quad (A\sigma)\tau = A(\sigma\tau) \quad a1=a \quad (a\sigma)\tau=a(\sigma\tau)
    \]

  \item[$\bullet$] There is also the operation of \emph{context extension}: if
    $\vdash \Gamma~ctxt$ and $\Gamma \vdash A~type$, there is a context
    $\Gamma, x:A$ (for brevity also written as $\Gamma.A$ from now on), a
    context morphism $\mathbf{p}: \Gamma.A \to \Gamma$, and a term $\Gamma.A
    \vdash \mathbf{q} : A \mathbf{p}$. This should satisfy the following
    property: for each $\vdash \Delta~ctxt$ and a context morphism
    $\sigma \from \Delta \to \Gamma$ and $\Delta \vdash a : A \sigma$, there is
    a context morphism $(\sigma, a): \Delta \to \Gamma.A$ such that:
    \[
      \mathbf{p}(\sigma, a) = \sigma \quad \mathbf{q}(\sigma, a) = a \quad (\sigma, a) \tau =
      (\sigma \tau, a \tau) \quad (\mathbf{p}, \mathbf{q}) = 1
    \]
    Where in the last equation $1: \Gamma.A \to \Gamma.A$
\end{itemize}

\section*{Presheaf model}

Let $\cat{C}$ be a small category. In the following objects of $\cat{C}$ will
be denoted as $I,J,K$ and morphisms as $f,g,h$.

The context $\Gamma$ is then a presheaf on $\cat{C}$, $\Gamma \from
\cat{C}^\op \to \Set$, that is, we are given a set $\Gamma(I)$ for each
$I$ in $\ob(\cat{C})$, and functions $\Gamma(I) \to \Gamma(J)$, $\rho \mapsto \rho
f$ (called \emph{restriction} and written as $f$ acting on the right) for each
$f: J \to I$ such that
\[
  \rho 1 = \rho \quad \text{and} \quad (\rho f) g = \rho (f g)
\]
for $g: K \to J$ and $f: J \to I$. We write $\Gamma_f$ for $\rho \mapsto \rho
f$.

The empty context then is a terminal presheaf and a context morphism $\sigma
\from \Delta \to \Gamma$ is a natural transformation making the diagram
commute.

\[
  \xymatrix{
    \Delta(I) \ar[r]^{\sigma_I} \ar[d]^{\Delta_f} & \Gamma(I) \ar[d]^{\Gamma_f} \\
    \Delta(J) \ar[r]^{\sigma_J} & \Gamma(J)
  }
\]

If we write $\Gamma_f$ and $\Delta_f$ as $f$ acting on the right, the
equation becomes:
\[
  (\sigma \rho) f = \sigma (\rho f)
\]
for $\rho$ in $\Delta(I)$.

Now we consider how to give a dependent type $\Gamma \vdash A~type$ in context
$\Gamma$. For each object $I \in \ob(\cat{C})$ and $\rho \in \Gamma(I)$ we require a
set $A \rho$, and for each $f \from J \to I$, we require a function $A \rho \to
A (\rho f)$, written as $a \mapsto af$ satisfying $a1 = a$ and $afg = a (f g)$
if $g \from K \to J$. The dependence on $I$ in $A\rho$ and on $I$ and $\rho$ in
$af$ is omitted for brevity. Substitution $\Delta \vdash A \sigma$ with $\sigma
\from \Delta \to \Gamma$ is simply given $(A \sigma) \rho = A (\sigma \rho)$
for $\rho \in \Gamma(I)$, together with the induced map
\[
  (A\sigma)\rho = A (\sigma \rho) \to A ((\sigma \rho)f) = A (\sigma (\rho f)) = (A \sigma) (\rho f)
\]
for $f \from J \to I$.

The definition of a dependent type can be rephrased more categorically: $A$ is
a presheaf on $\int_C \Gamma$, where $\int_C \Gamma$ is a category of elements
of the presheaf $\Gamma$, defined as follows:
\begin{itemize}
  \item objects are pairs $(I, \rho)$, where $I \in \ob(\cat{C})$ and $\rho \in
    \Gamma(I)$
  \item a morphism $(J, \rho') \to (I, \rho)$ is a morphism $f \from J \to I$
    in $\cat{C}$ such that $\rho' = \Gamma_f \rho$
\end{itemize}

A term $\Gamma \vdash a : A$ of a dependent type $\Gamma \vdash A~type$ is
given by a family of elements $a \rho \in A \rho$ for each $I \in \ob(\cat{C})$
and $\rho \in \Gamma(I)$, such that $a \rho f = a (\rho f)$ for each $f \from J
\to I$. The substitution $\Delta \vdash a \sigma : A \sigma$ with $\sigma \from
\Delta \to \Gamma$ is given by the family $(a \sigma) \rho = a (\sigma \rho)$.

For $\Gamma \vdash A~type$ the context extension $\vdash \Gamma.A~ctxt$ is
defined by
\begin{align*}
  &(\Gamma.A)(I) = \{(\rho, u)~|~\rho \in \Gamma(I) \land u \in A \rho\} &&
  \text{for $I \in \cat{C}$}\\
  &(\rho, u)f = (\rho f, u f) && \text{for $f \from J \to I$}
\end{align*}
and the projections are defined by $\mathbf{p} \from \Gamma.A \to \Gamma$,
$\mathbf{p}(\rho, u) = \rho$, and $\Gamma.A \vdash \mathbf{q} : Ap$ by
$\mathbf{q}(\rho, u) = u$.  Now assume $\sigma \from \Delta \to \Gamma$ and
$\Delta \vdash a : A\sigma$; we define $(\sigma, a) \from \Delta \to \Gamma.A$
by $(\sigma, a)\rho = (\sigma \rho, a \rho)$.

\subsection*{Dependent product}

As a motivation for dependent products recall how to define exponents in
presheaf categories. Let $\Gamma$ and $\Delta$ be presheaves and suppose we
already know how the exponent $\Delta^\Gamma$ is constructed. Then we get by
the Yoneda Lemma and the fact that $-^\Gamma$ should be right adjoint to $-
\times \Gamma$, that
\[
  (\Delta^\Gamma)(I) \cong Hom(\mathbf{y}I, \Delta^\Gamma) \cong Hom(\mathbf{y}I \times \Gamma, \Delta)
\]
where $\mathbf{y}$ is Yoneda embedding.

The latter can be taken up as a definition. So an element $w$ of
$(\Delta^\Gamma)(I)$ is a natural transformation $w \from \mathbf{y}I\times
\Gamma \to \Delta$, so it is given by functions
\[
  w_J \from (\mathbf{y}I)(J) \times \Gamma(J) \to \Delta(J),
\]
and the naturality condition becomes $(w_J(f, \rho))g = w_K(fg, \rho g)$ for $f
\from J \to I$, $\rho \in \Gamma(J)$, and $g \from K \to J$.

We will now show that CwF associated to a presheaf category supports
$\Pi$-types. Given $\Gamma \vdash A$ and $\Gamma.A \vdash B$ we have to define
$\Gamma \vdash \Pi x: A. B~type$, that is, we have to define the set $(\Pi x:
A.B) \rho$ for $I \in \cat{C}$ and $\rho \in \Gamma(I)$. The elements $w$ of
$(\Pi x:A.B)\rho$ are families $w = (w_f~|~J\in \cat{C}, f \from J \to I)$ of
(dependent) functions such that
\[
  w_f u \in B(\rho f, u) \qquad \text{for $J \in \cat{C}, f \from J \to I, and u \in A(\rho f)$},
\]
with the requirement that for $g \from K \to J$
\[
  (w_f u)g = w_{fg}(ug).
\]

For such a family $w \in (\Pi x: A. B)\rho$, the restriction $wf \in (\Pi
x:A.B)(\rho f)$ for $f: \from J \to I$ is defined by taking
\[
  (w f)_g u = w_{fg}u \in B(\rho f g, u)
\]
where $g \from K \to J$ and $u \in A(\rho f g)$. This definition satisfies $w 1
= w$ and $w f g = w(f g)$.

Now let $\Gamma.A \vdash b : B$; we have to define $\Gamma \vdash \lambda b :
\Pi x:A.B$, that is , give $(\lambda b)\rho \in (\Pi x:A.B)\rho$. For $f \from
J \to I$ and $u \in A(\rho f)$ we set
\[
  ((\lambda b)\rho)_f u = b (\rho f, u) \in B(\rho f, u)
\]
This satisfies for $g \from K \to J$
\[
  (((\lambda b)\rho)_f u)g = (b(\rho f, u))g = b(\rho(fg), ug) = ((\lambda b)\rho)_{fg}(ug)
\]
and defines a term since
\[
  (((\lambda b)\rho)f)_g u = ((\lambda b) \rho)_{fg} u = b(\rho fg, u) = ((\lambda b)(\rho f))_g u.
\]

To define the application let $\Gamma \vdash u : \Pi x:A.B$ and $\Gamma \vdash
v : A$, and we set $App(u, v)\rho = (u\rho)_1 (v\rho) \in B(\rho, v\rho)$, thus
$App(u, v)\rho \in B(1, v)\rho$ as $(\rho, v \rho) = (1, v)\rho$.
$\beta$-equality is readily checked
\[
  App(\lambda b, v)\rho = ((\lambda b)\rho)_1(v\rho)=b(\rho, v\rho)
\]
and similarly for $\eta$-equality and other equations.

\begin{example}[Ordinary function type]
  Let us consider a term of type $A \to A$, and the elements in $Ty(\Gamma)$
  and $Ter(\Gamma, A \to A)$ for some $\vdash \Gamma~ctxt$, $\Gamma \vdash
  A~type$.

  For a CwF $(\cat{C}, F), I, J \in ob(\cat{C}), f \from J \to I$ we give the
  element of $\mathit{Ty}(\Gamma)$ corresponding to the type $A \to A$:
  \[
    (A \to A)\rho = \{w | w_f u \in A(\rho f), u \in A(\rho f)\} = \{\lambda I f u.u\}
  \]
  The required equalities hold trivially for any $g \from K \to J$, $((\lambda
  I f u.u)_f u) g = (\lambda I f u.u)_{fg} (ug)$.

  $\Gamma, x : A \vdash x : A$ corresponds to $q$ in the model. Thus $\Gamma
  \vdash \lambda x.x : A -> A$ corresponds to $\lambda q$ being defined as
  follows:
  \[
    ((\lambda q) \rho)_f u = q(\rho f, u) = u
  \]
\end{example}

\subsection*{Dependent sum}
The interpretation of $\Gamma \vdash \Sigma x:A.B~type$ for $\Gamma \vdash
A~type$ and $\Gamma.A \vdash B~type$ is defined by
\[
  (\Sigma x :A. B)\rho = \{(a, b)~|~a \in A\rho \land b \in B(\rho, a)\}
\]
for $\rho \in \Gamma(I), I \in \ob(\cat{C})$. The restrictions are defined
componentwise $(a, b)f = (af, bf)$ for $f \from J \to I$. The pairing operation
$\Gamma \vdash (u, v) : \Sigma x:A.B$ of $\Gamma \vdash u : A$ and $\Gamma.A
\vdash v : B(1, u)$ is defined by the componentwise pairing: $(u, v)\rho =
(u\rho, v\rho)$. Likewise for the projections $\mathbf{p}$ and $\mathbf{q}$,
$(\mathbf{p}w)\rho = a$ and $(\mathbf{q}w)\rho = b$ for $w\rho = (a, b)$. This
validates the necessary equations.

\section*{Conclusion}

\end{document}

% vim: ts=2 ss=2 sw=2 expandtab
